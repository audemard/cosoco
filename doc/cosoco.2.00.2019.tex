\documentclass{llncs}

%\usepackage{makeidx}  % allows for indexgeneration

\usepackage{mathrsfs} % for mathscr
\usepackage{algorithmic}
\usepackage{algorithm}

\usepackage{amsmath}
\usepackage{amssymb}

\usepackage{color}
\usepackage{subfigure}
\usepackage[table]{xcolor}
\usepackage{graphicx}
\usepackage{hyperref}

\hypersetup{colorlinks,linkcolor={red!50!black},citecolor={blue!50!black},urlcolor={blue!80!black}}

\newcommand{\f}[1]{\mathtt{#1}} %\tt #1} % field
\newcommand{\h}[1]{\textit{#1}} % heuristic
\newcommand{\p}[1]{\textit{#1}} % problem

%\ifx\pdftexversion\undefined
%  \usepackage[dvips]{graphics}
%\else
%  \usepackage[pdftex]{graphics}
%\fi

%\newcommand{\sub}[1]{\xrightarrow[ _{#1}]{}}

\begin{document}

%\pagestyle{headings}  % switches on printing of running heads
\pagestyle{empty}


\title{CoSoCo 2.00\\ {\small XCSP3 Competition 2010}}

\author{Gilles Audemard}

\institute{CRIL-CNRS, UMR 8188\\
Universit\'e d'Artois, F-62307 Lens France\\
\email{audemard@cril.fr}\\
}
%\date{03 September 2017}

%\begin{document}

\maketitle




\bigskip\bigskip\bigskip CoSoCo is a constraint solver written in
C++.The main goal is to build
a simple, but efficient constraint solver. Indeed, the main part of
the solver contains less than 2,000 lines of code. CoSoCo will be
available on bitbucket as soon as possible. CoSoCo recognizes XCSP3
\cite{BLPP_xcsp3} by using the C++ parser that can be downloaded at
\href{https://github.com/xcsp3team/XCSP3-CPP-Parser}{https://github.com/xcsp3team/XCSP3-CPP-Parser}. It
can deal with almost all XCSP3 Core constraints. The part related to
all constraint propagators contains around 4,500 lines of codes.

 This is the third participation of CoSoCo to XCSP competitions. 
This year CoSoCo contains some new features: 
\begin{itemize}
\item A parallel version built on top of the pFactory library
  \cite{pfactory}. You can also take a look to \url{https://github.com/crillab/pfactory}.
\item Transformation of intension constraints to table constraints.
\item Some additional constraints are supported: circuit, No-Overlap,
  N-Values ...
\end{itemize}

\bigskip
CoSoCo performs backtrack search, enforcing (generalized) arc consistency at each node (when possible).  
The main components are :
\begin{itemize}
\item \h{dom/wdeg} \cite{BHLS_boosting} as variable ordering heuristic;
\item \h{lexico} as value ordering heuristic;
\item lc(1), last-conflict reasoning with a collecting parameter $k$ set to $1$, as described in \cite{LSTV_reasonning};
\item a geometric restart policiy;
\item a variable-oriented propagation scheme \cite{G_relational}, where a queue $Q$ records all variables with recently reduced domains (see Chapter 4 in \cite{L_constraint}).
\end{itemize}
  


\section*{Acknowledgements}
This work would not have taken place without Christophe
Lecoutre. I  would like to thank him  very warmly for his support.


\bibliographystyle{plain} %alpha}
\bibliography{globalBiblio}


\end{document}

